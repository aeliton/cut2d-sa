% Use only LaTeX2e, calling the article.cls class and 12-point type.

\documentclass[12pt]{article}

\usepackage{caption}
\usepackage{subcaption}
\usepackage{multirow}


% Users of the {thebibliography} environment or BibTeX should use the
% scicite.sty package, downloadable from *Science* at
% www.sciencemag.org/about/authors/prep/TeX_help/ .
% This package should properly format in-text
% reference calls and reference-list numbers.

\usepackage{scicite}
\usepackage[pdftex]{graphicx}
\usepackage{amsthm}
\usepackage{mathtools}
\usepackage[utf8]{inputenc}
\usepackage[T1]{fontenc}
% Use times if you have the font installed; otherwise, comment out the
% following line.

\usepackage{times}

% The preamble here sets up a lot of new/revised commands and
% environments.  It's annoying, but please do *not* try to strip these
% out into a separate .sty file (which could lead to the loss of some
% information when we convert the file to other formats).  Instead, keep
% them in the preamble of your main LaTeX source file.
\usepackage[utf8]{inputenc}

% Default fixed font does not support bold face
\DeclareFixedFont{\ttb}{T1}{txtt}{bx}{n}{12} % for bold
\DeclareFixedFont{\ttm}{T1}{txtt}{m}{n}{12}  % for normal

% Custom colors
\usepackage{color}
\definecolor{deepblue}{rgb}{0,0,0.5}
\definecolor{deepred}{rgb}{0.6,0,0}
\definecolor{deepgreen}{rgb}{0,0.5,0}

\usepackage{listings}

% Python style for highlighting
\newcommand\pythonstyle{\lstset{
language=Python,
basicstyle=\ttm,
otherkeywords={self},             % Add keywords here
keywordstyle=\ttb\color{deepblue},
emph={MyClass,__init__},          % Custom highlighting
emphstyle=\ttb\color{deepred},    % Custom highlighting style
stringstyle=\color{deepgreen},
frame=tb,                         % Any extra options here
showstringspaces=false            % 
}}


% Python environment
\lstnewenvironment{python}[1][]
{
\pythonstyle
\lstset{#1}
}
{}

% Python for external files
\newcommand\pythonexternal[2][]{{
\pythonstyle
\lstinputlisting[#1]{#2}}}

% Python for inline
\newcommand\pythoninline[1]{{\pythonstyle\lstinline!#1!}}


% The following parameters seem to provide a reasonable page setup.

\topmargin 0.0cm
\oddsidemargin 0.2cm
\textwidth 16cm 
\textheight 21cm
\footskip 1.0cm


%The next command sets up an environment for the abstract to your paper.

\newenvironment{sciabstract}{%
\begin{quote} \bf}
{\end{quote}}


% If your reference list includes text notes as well as references,
% include the following line; otherwise, comment it out.

\renewcommand\refname{Referências e Notas}

% The following lines set up an environment for the last note in the
% reference list, which commonly includes acknowledgments of funding,
% help, etc.  It's intended for users of BibTeX or the {thebibliography}
% environment.  Users who are hand-coding their references at the end
% using a list environment such as {enumerate} can simply add another
% item at the end, and it will be numbered automatically.

\newcounter{lastnote}
\newenvironment{scilastnote}{%
\setcounter{lastnote}{\value{enumiv}}%
\addtocounter{lastnote}{+1}%
\begin{list}%
{\arabic{lastnote}.}
{\setlength{\leftmargin}{.22in}}
{\setlength{\labelsep}{.5em}}}
{\end{list}}


% Include your paper's title here

\title{Aplicação de {\it Simulated Anealing\/} Para o Problema de Corte 2D Guilhotinado}

% Place the author information here.  Please hand-code the contact
% information and notecalls; do *not* use \footnote commands.  Let the
% author contact information appear immediately below the author names
% as shown.  We would also prefer that you don't change the type-size
% settings shown here.

\author
{Aeliton G. Silva,$^{1}$ Alano Martins,$^{1}$\\
\\
\normalsize{$^{1}$Mestrado Academico em Ciência da Computação, Universidade Estadual do Ceará,}\\
\normalsize{Av. Dr. Silas Munguba, 1700, Campus do Itaperi, Fortaleza-CE, Brasil}\\
\\
%\normalsize{$^\ast$To whom correspondence should be addressed; E-mail:  jsmith@wherever.edu.}
}

% Include the date command, but leave its argument blank.

\date{}



%%%%%%%%%%%%%%%%% END OF PREAMBLE %%%%%%%%%%%%%%%%



\begin{document} 

% Double-space the manuscript.

\baselineskip24pt

% Make the title.

\maketitle 



% Place your abstract within the special {sciabstract} environment.

\begin{sciabstract}
    Este documento apresenta a aplicação da metaheurística \textit{Simulated
    Anealing} ao problema de Corte Bidmensional Guilhotinado. Provemos os
    resultados da execução de 10 instâncias, cada uma executada 3 vezes e
    selecionada a melhor delas para apresentação em uma tabela com todos os
    melhores resultados obtidos. Apresentamos também os métodos mais
    importantes da implementação, em python, de nossa solução.
\end{sciabstract}

% In setting up this template for *Science* papers, we've used both
% the \section* command and the \paragraph* command for topical
% divisions.  Which you use will of course depend on the type of paper
% you're writing.  Review Articles tend to have displayed headings, for
% which \section* is more appropriate; Research Articles, when they have
% formal topical divisions at all, tend to signal them with bold text
% that runs into the paragraph, for which \paragraph* is the right
% choice.  Either way, use the asterisk (*) modifier, as shown, to
% suppress numbering.

\section*{Introdução}

O corte guilhotinado 2D para objetos retangulares é um desafio para otimização do aproveitamento dessas peças, reduzindo-as em objetos menores com o maior aproveitamento, ou seja, menor perda de materiais restantes após os cortes. 
Apesar do facil entendimento da problemática, o corte guilhotinável 2D é considerado um problema \textit{NP-HARD}, o qual não há algoritmos deterministicos para resolução em tempo hábil para instâncias de médio ou grande porte.
Esse problema é dado por uma grande variedade de arranjos possíveis, tornando inviável uso de métodos exatos. Assim os métodos heuristicos e meta heuristicos são uma boa solução.
Esse trabalho visa demonstrar uma implementação para o problema de corte guilhotinado 2D utilizando a meta heuristica \textit{Simulated Annealing}. 

\section*{Definições}
	
	Diversos processos produtivos passam por uma atividade de corte em diversos tipos de materiais, afim de reduzir uma peça maior em várias outras menores com a finalidade de atender demanda por itens especificos, porém reduzindo os disperdicios do material original. 
	Se um corte numa área retangular resulta em outros dois retangulos, esse problema é considerado guilhotinável ortogonal e quando isso não ocorre, guilhotinável não ortogonal. Para restringir o problema, os cortes devem ser realizados paralelos ao lado dos retângulos como demonstrado na figura:
	
	\begin{figure}[h]
    \centering
    \includegraphics[width=15cm, height=5cm]{imagens/guilhotine2}
    \caption{Exemplos de cortes}
  \end{figure}
	
		Segundo   O problema para cortes em peças limitadas pelo tamanho do objeto origianal e maior, realizando apenas cortes guilhotináveis e não estagiados, esse problema também é conhecido como  Problema de Corte Bidimensional Guilhotinado e Restrito.
	

	

\section*{Perturbação e Vizinhança}

Following are a few things to keep in mind in coding equations,
tables, and figures for submission to {\it Science}.

\paragraph*{In-line math.}  The utility that we use for converting
from \LaTeX\ to HTML handles in-line math relatively well.  It is best
to avoid using built-up fractions in in-line equations, and going for
the more boring ``slash'' presentation whenever possible --- that is,
for \verb+$a/b$+ (which comes out as $a/b$) rather than
\verb+$\frac{a}{b}$+ (which compiles as $\frac{a}{b}$).  Likewise,
HTML isn't tooled to handle certain overaccented special characters
in-line; for $\hat{\alpha}$ (coded \verb+$\hat{\alpha}$+), for
example, the HTML translation code will return [\^{}$(\alpha)$].
Don't drive yourself crazy --- but if it's possible to avoid such
constructs, please do so.  Please do not code arrays or matrices as
in-line math; display them instead.  And please keep your coding as
\TeX-y as possible --- avoid using specialized math macro packages
like \texttt{amstex.sty}.

\paragraph*{Displayed math.} Our HTML converter sets up \TeX\
displayed equations using nested HTML tables.  That works well for an
HTML presentation, but Word chokes when it comes across a nested
table in an HTML file.  We surmount that problem by simply cutting the
displayed equations out of the HTML before it's imported into Word,
and then replacing them in the Word document using either images or
equations generated by a Word equation editor.  Strictly speaking,
this procedure doesn't bear on how you should prepare your manuscript
--- although, for reasons best consigned to a note \cite{nattex}, we'd
prefer that you use native \TeX\ commands within displayed-math
environments, rather than \LaTeX\ sub-environments.

\paragraph*{Tables.}  The HTML converter that we use seems to handle
reasonably well simple tables generated using the \LaTeX\
\texttt{\{tabular\}} environment.  For very complicated tables, you
may want to consider generating them in a word processing program and
including them as a separate file.

\paragraph*{Figures.}  Figure callouts within the text should not be
in the form of \LaTeX\ references, but should simply be typed in ---
that is, \verb+(Fig. 1)+ rather than \verb+\ref{fig1}+.  For the
figures themselves, treatment can differ depending on whether the
manuscript is an initial submission or a final revision for acceptance
and publication.  For an initial submission and review copy, you can
use the \LaTeX\ \verb+{figure}+ environment and the
\verb+\includegraphics+ command to include your PostScript figures at
the end of the compiled PostScript file.  For the final revision,
however, the \verb+{figure}+ environment should {\it not\/} be used;
instead, the figure captions themselves should be typed in as regular
text at the end of the source file (an example is included here), and
the figures should be uploaded separately according to the Art
Department's instructions.


\section*{Meta-Heurística \textit{Simulated Anealing}}

A meta heuristica Simulated Annealing (S.A) é uma técnica oriunda de uma analogia com a termodinamica para obter estados de baixas energias em um sólido. Como meta heuristica, ela é utilizada em problemas de otimização de busca local com probabilidade. 
Inicialmente o S.A a busca a partir de um indice qualquer, iniciando um laço em que cada interação procura um candidato seguinte para um ponto minimo na vinzinhança do atual candidato. Essa forma é dada pela seguinte diferença entre as funções objetivas:

$ {\triangle = f (s^1) - f(s) }$

Logo se $ {\triangle < 0 }$, houve uma diminuição na função objetiva e $ s^1$ é considerava como nova solução. Caso $ {\triangle = 0 }$ ou $ {\triangle > 0 }$ houve um aumento ou estabilidade da energia e a aceitação dessa solução será feita sobre um fator estatistico onde é mais provável para autas temperaturas e menos provável para menores. O fator utilizado é conhecido como fator de Boltzmann, dado por: $ {e^{(\triangle / T)} }$ onde T é a temperatura que regula a aceitação da solução de pior custo.

Essa implementação é vista no código fonte em:

\begin{python}
        def execute(self, start):
        solution = start
        temperature = self.__initial_temperature(solution)
        success_iterator = 0
        temperatures = []
        costs = []
        solutions = []

        no_change_counter = 0
        for j in range(0, self.MAX_INTERATIONS):
            for i in range(0, self.MAX_RANDOMIZE):
                new_solution = self.__randomize(solution)
                diff_s = self.__diff_solution(new_solution, solution)

                try:
                    if diff_s >= 0 or math.exp(-diff_s / temperature) > uniform(0, 1):
                        solution = new_solution
                        success_iterator = success_iterator + 1
                        
                except:
                    pass

                if success_iterator >= self.MAX_SUCESS:  
                    break

            solutions.insert(j, self.__cost(solution))
            if j > 0 and solutions[j] == solutions[j-1]:
                no_change_counter += 1
            else:
                no_change_counter = 0

            temperatures.append(temperature)
            costs.append(self.__cost(solution))

            temperature = self.ALPHA * temperature

            if success_iterator == 0 or 0.2 * self.MAX_INTERATIONS <= no_change_counter:  # stop condition
                break

        return temperatures, costs, solution
\end{python}

A temperatura inicia com um valor alto, mas após uma quantidade fixa de interações ($MAX\_INTERATIONS$) vai gradativamente caindo em razão de um fator, ALPHA, dessa forma o algoritmo tende a escapar de ficar preso em mínimos locais. Com a aproximação da temperatura a 0, o algoritmo se comporta como método de descida, já que a probabilidade de aceitar funções que piorem é muito baixa.
O algoritmo encontra sua condição de parada se a temperatura chegar próxima a 0 ou nenhuma nova solução aceita após várias interações, ou seja, o sistema está estável.

\section*{Execução dos Testes}


\begin{center}
    \resizebox{\textwidth}{!}{%
    \begin{tabular}{ |c|c|c|c|c|c|c|c|c|}
    \hline
        \# & Instancia & \# de Itens & Tam. Placa & S. Inicial & S. Média & S. Melhor & Desperdício & Tempo (s) \\ \hline

1 & cut12 & 50 & 1000x1000 & 245443 & 26131955 & 960379 & 3.96 & 1.505860 \\ \hline
2 & cut13 & 32 & 3000x3000 & 1819609 & 33351152 & 8296226 & 7.82 & 7.122655 \\ \hline
3 & cut14 & 34 & 3000x3000 & 1160710 & 20266030 & 8443673 & 6.18 & 13.024269 \\ \hline
4 & cut17 & 82 & 3500x3500 & 3063289 & 40185532 & 11142932 & 9.04 & 8.674745 \\ \hline
5 & cut1 & 10 & 250x250 & 35358 & 1416513 & 58136 & 6.98 & 1.410658 \\ \hline
6 & cut2 & 20 & 250x250 & 26941 & 1265506 & 53357 & 14.63 & 1.742492 \\ \hline
7 & cut5 & 10 & 500x500 & 191038 & 7183387 & 219110 & 12.36 & 1.299100 \\ \hline
8 & cut7 & 30 & 500x500 & 83885 & 5937212 & 244655 & 2.14 & 1.472742 \\ \hline
9 & cut9 & 10 & 1000x1000 & 299700 & 24435095 & 953628 & 4.64 & 1.634113 \\ \hline
10 & livre & 82 & 3500x3500 & 2972333 & 25909328 & 11002738 & 10.18 & 14.329081 \\ \hline

    \end{tabular}}
\end{center}



\begin{figure}
\centering
\begin{subfigure}{.5\textwidth}
  \centering
  \includegraphics[width=1\linewidth]{results/cut12/3/plot}
  \label{fig:sub1}
\end{subfigure}%
\begin{subfigure}{.5\textwidth}
  \centering
  \includegraphics[width=1\linewidth]{results/cut12/3/cut}
  \label{fig:sub2}
\end{subfigure}
\caption{Instancia cut12.txt, 11.03\% 50 1000x1000 470292 23819525 889729 1.502228}
\label{fig:test}
\end{figure}


\begin{figure}
\centering
\begin{subfigure}{.5\textwidth}
  \centering
  \includegraphics[width=1\linewidth]{results/cut13/1/plot}
  \label{fig:sub1}
\end{subfigure}%
\begin{subfigure}{.5\textwidth}
  \centering
  \includegraphics[width=1\linewidth]{results/cut13/1/cut}
  \label{fig:sub2}
\end{subfigure}
\caption{Instancia cut13.txt, 7.74\% 32 3000x3000 1414880 38189024 8303664 6.923704}
\label{fig:test}
\end{figure}


\begin{figure}
\centering
\begin{subfigure}{.5\textwidth}
  \centering
  \includegraphics[width=1\linewidth]{results/cut14/2/plot}
  \label{fig:sub1}
\end{subfigure}%
\begin{subfigure}{.5\textwidth}
  \centering
  \includegraphics[width=1\linewidth]{results/cut14/2/cut}
  \label{fig:sub2}
\end{subfigure}
\caption{Instancia cut14.txt, 4.20\% 34 3000x3000 1887643 32370005 8622202 6.011851}
\label{fig:test}
\end{figure}


\begin{figure}
\centering
\begin{subfigure}{.5\textwidth}
  \centering
  \includegraphics[width=1\linewidth]{results/cut17/1/plot}
  \label{fig:sub1}
\end{subfigure}%
\begin{subfigure}{.5\textwidth}
  \centering
  \includegraphics[width=1\linewidth]{results/cut17/1/cut}
  \label{fig:sub2}
\end{subfigure}
\caption{Instancia cut17.txt, 6.51\% 82 3500x3500 1556649 38985507 11453057 9.449118}
\label{fig:test}
\end{figure}


\begin{figure}
\centering
\begin{subfigure}{.5\textwidth}
  \centering
  \includegraphics[width=1\linewidth]{results/cut1/1/plot}
  \label{fig:sub1}
\end{subfigure}%
\begin{subfigure}{.5\textwidth}
  \centering
  \includegraphics[width=1\linewidth]{results/cut1/1/cut}
  \label{fig:sub2}
\end{subfigure}
\caption{Instancia cut1.txt, 9.95\% 10 250x250 30728 1353903 56284 1.414139}
\label{fig:test}
\end{figure}


\begin{figure}
\centering
\begin{subfigure}{.5\textwidth}
  \centering
  \includegraphics[width=1\linewidth]{results/cut2/2/plot}
  \label{fig:sub1}
\end{subfigure}%
\begin{subfigure}{.5\textwidth}
  \centering
  \includegraphics[width=1\linewidth]{results/cut2/2/cut}
  \label{fig:sub2}
\end{subfigure}
\caption{Instancia cut2.txt, 7.63\% 20 250x250 15960 1552967 57731 1.476343}
\label{fig:test}
\end{figure}


\begin{figure}
\centering
\begin{subfigure}{.5\textwidth}
  \centering
  \includegraphics[width=1\linewidth]{results/cut5/2/plot}
  \label{fig:sub1}
\end{subfigure}%
\begin{subfigure}{.5\textwidth}
  \centering
  \includegraphics[width=1\linewidth]{results/cut5/2/cut}
  \label{fig:sub2}
\end{subfigure}
\caption{Instancia cut5.txt, 11.74\% 10 500x500 114594 5644342 220652 1.622015}
\label{fig:test}
\end{figure}


\begin{figure}
\centering
\begin{subfigure}{.5\textwidth}
  \centering
  \includegraphics[width=1\linewidth]{results/cut7/3/plot}
  \label{fig:sub1}
\end{subfigure}%
\begin{subfigure}{.5\textwidth}
  \centering
  \includegraphics[width=1\linewidth]{results/cut7/3/cut}
  \label{fig:sub2}
\end{subfigure}
\caption{Instancia cut7.txt, 4.98\% 30 500x500 89116 5985148 237543 1.515644}
\label{fig:test}
\end{figure}


\begin{figure}
\centering
\begin{subfigure}{.5\textwidth}
  \centering
  \includegraphics[width=1\linewidth]{results/cut9/2/plot}
  \label{fig:sub1}
\end{subfigure}%
\begin{subfigure}{.5\textwidth}
  \centering
  \includegraphics[width=1\linewidth]{results/cut9/2/cut}
  \label{fig:sub2}
\end{subfigure}
\caption{Instancia cut9.txt, 4.64\% 10 1000x1000 299700 24677062 953628 1.532539}
\label{fig:test}
\end{figure}



\bibliography{scibib}

\bibliographystyle{Science}



% Following is a new environment, {scilastnote}, that's defined in the
% preamble and that allows authors to add a reference at the end of the
% list that's not signaled in the text; such references are used in
% *Science* for acknowledgments of funding, help, etc.

\begin{scilastnote}
\item We've included in the template file \texttt{scifile.tex} a new
environment, \texttt{\{scilastnote\}}, that generates a numbered final
citation without a corresponding signal in the text.  This environment
can be used to generate a final numbered reference containing
acknowledgments, sources of funding, and the like, per {\it Science\/}
style.
\end{scilastnote}




% For your review copy (i.e., the file you initially send in for
% evaluation), you can use the {figure} environment and the
% \includegraphics command to stream your figures into the text, placing
% all figures at the end.  For the final, revised manuscript for
% acceptance and production, however, PostScript or other graphics
% should not be streamed into your compliled file.  Instead, set
% captions as simple paragraphs (with a \noindent tag), setting them
% off from the rest of the text with a \clearpage as shown  below, and
% submit figures as separate files according to the Art Department's
% instructions.


\clearpage

\noindent {\bf Fig. 1.} Please do not use figure environments to set
up your figures in the final (post-peer-review) draft, do not include graphics in your
source code, and do not cite figures in the text using \LaTeX\
\verb+\ref+ commands.  Instead, simply refer to the figure numbers in
the text per {\it Science\/} style, and include the list of captions at
the end of the document, coded as ordinary paragraphs as shown in the
\texttt{scifile.tex} template file.  Your actual figure files should
be submitted separately.



\end{document}
